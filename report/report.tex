
%% bare_jrnl_compsoc.tex
%% V1.4
%% 2012/12/27
%% by Michael Shell
%% See:
%% http://www.michaelshell.org/
%% for current contact information.
%%
%% This is a skeleton file demonstrating the use of IEEEtran.cls
%% (requires IEEEtran.cls version 1.8 or later) with an IEEE Computer
%% Society journal paper.
%%
%% Support sites:
%% http://www.michaelshell.org/tex/ieeetran/
%% http://www.ctan.org/tex-archive/macros/latex/contrib/IEEEtran/
%% and
%% http://www.ieee.org/

%
\documentclass[12pt,journal,compsoc]{IEEEtran}
%
% If IEEEtran.cls has not been installed into the LaTeX system files,
% manually specify the path to it like:
% \documentclass[12pt,journal,compsoc]{../sty/IEEEtran}





% Some very useful LaTeX packages include:
% (uncomment the ones you want to load)


% *** MISC UTILITY PACKAGES ***
%
%\usepackage{ifpdf}
% Heiko Oberdiek's ifpdf.sty is very useful if you need conditional
% compilation based on whether the output is pdf or dvi.
% usage:
% \ifpdf
%   % pdf code
% \else
%   % dvi code
% \fi
% The latest version of ifpdf.sty can be obtained from:
% http://www.ctan.org/tex-archive/macros/latex/contrib/oberdiek/
% Also, note that IEEEtran.cls V1.7 and later provides a builtin
% \ifCLASSINFOpdf conditional that works the same way.
% When switching from latex to pdflatex and vice-versa, the compiler may
% have to be run twice to clear warning/error messages.






% *** CITATION PACKAGES ***
%
\ifCLASSOPTIONcompsoc
  % IEEE Computer Society needs nocompress option
  % requires cite.sty v4.0 or later (November 2003)
  % \usepackage[nocompress]{cite}
\else
  % normal IEEE
  % \usepackage{cite}
\fi
% cite.sty was written by Donald Arseneau
% V1.6 and later of IEEEtran pre-defines the format of the cite.sty package
% \cite{} output to follow that of IEEE. Loading the cite package will
% result in citation numbers being automatically sorted and properly
% "compressed/ranged". e.g., [1], [9], [2], [7], [5], [6] without using
% cite.sty will become [1], [2], [5]--[7], [9] using cite.sty. cite.sty's
% \cite will automatically add leading space, if needed. Use cite.sty's
% noadjust option (cite.sty V3.8 and later) if you want to turn this off
% such as if a citation ever needs to be enclosed in parenthesis.
% cite.sty is already installed on most LaTeX systems. Be sure and use
% version 4.0 (2003-05-27) and later if using hyperref.sty. cite.sty does
% not currently provide for hyperlinked citations.
% The latest version can be obtained at:
% http://www.ctan.org/tex-archive/macros/latex/contrib/cite/
% The documentation is contained in the cite.sty file itself.
%
% Note that some packages require special options to format as the Computer
% Society requires. In particular, Computer Society  papers do not use
% compressed citation ranges as is done in typical IEEE papers
% (e.g., [1]-[4]). Instead, they list every citation separately in order
% (e.g., [1], [2], [3], [4]). To get the latter we need to load the cite
% package with the nocompress option which is supported by cite.sty v4.0
% and later. Note also the use of a CLASSOPTION conditional provided by
% IEEEtran.cls V1.7 and later.


\usepackage[utf8]{inputenc}
\usepackage{url}


% *** GRAPHICS RELATED PACKAGES ***
%
\ifCLASSINFOpdf
  % \usepackage[pdftex]{graphicx}
  % declare the path(s) where your graphic files are
  % \graphicspath{{../pdf/}{../jpeg/}}
  % and their extensions so you won't have to specify these with
  % every instance of \includegraphics
  % \DeclareGraphicsExtensions{.pdf,.jpeg,.png}
\else
  % or other class option (dvipsone, dvipdf, if not using dvips). graphicx
  % will default to the driver specified in the system graphics.cfg if no
  % driver is specified.
  % \usepackage[dvips]{graphicx}
  % declare the path(s) where your graphic files are
  % \graphicspath{{../eps/}}
  % and their extensions so you won't have to specify these with
  % every instance of \includegraphics
  % \DeclareGraphicsExtensions{.eps}
\fi
% graphicx was written by David Carlisle and Sebastian Rahtz. It is
% required if you want graphics, photos, etc. graphicx.sty is already
% installed on most LaTeX systems. The latest version and documentation
% can be obtained at: 
% http://www.ctan.org/tex-archive/macros/latex/required/graphics/
% Another good source of documentation is "Using Imported Graphics in
% LaTeX2e" by Keith Reckdahl which can be found at:
% http://www.ctan.org/tex-archive/info/epslatex/
%
% latex, and pdflatex in dvi mode, support graphics in encapsulated
% postscript (.eps) format. pdflatex in pdf mode supports graphics
% in .pdf, .jpeg, .png and .mps (metapost) formats. Users should ensure
% that all non-photo figures use a vector format (.eps, .pdf, .mps) and
% not a bitmapped formats (.jpeg, .png). IEEE frowns on bitmapped formats
% which can result in "jaggedy"/blurry rendering of lines and letters as
% well as large increases in file sizes.
%
% You can find documentation about the pdfTeX application at:
% http://www.tug.org/applications/pdftex






% *** MATH PACKAGES ***
%
%\usepackage[cmex10]{amsmath}
% A popular package from the American Mathematical Society that provides
% many useful and powerful commands for dealing with mathematics. If using
% it, be sure to load this package with the cmex10 option to ensure that
% only type 1 fonts will utilized at all point sizes. Without this option,
% it is possible that some math symbols, particularly those within
% footnotes, will be rendered in bitmap form which will result in a
% document that can not be IEEE Xplore compliant!
%
% Also, note that the amsmath package sets \interdisplaylinepenalty to 10000
% thus preventing page breaks from occurring within multiline equations. Use:
%\interdisplaylinepenalty=2500
% after loading amsmath to restore such page breaks as IEEEtran.cls normally
% does. amsmath.sty is already installed on most LaTeX systems. The latest
% version and documentation can be obtained at:
% http://www.ctan.org/tex-archive/macros/latex/required/amslatex/math/





% *** SPECIALIZED LIST PACKAGES ***
%
%\usepackage{algorithmic}
% algorithmic.sty was written by Peter Williams and Rogerio Brito.
% This package provides an algorithmic environment fo describing algorithms.
% You can use the algorithmic environment in-text or within a figure
% environment to provide for a floating algorithm. Do NOT use the algorithm
% floating environment provided by algorithm.sty (by the same authors) or
% algorithm2e.sty (by Christophe Fiorio) as IEEE does not use dedicated
% algorithm float types and packages that provide these will not provide
% correct IEEE style captions. The latest version and documentation of
% algorithmic.sty can be obtained at:
% http://www.ctan.org/tex-archive/macros/latex/contrib/algorithms/
% There is also a support site at:
% http://algorithms.berlios.de/index.html
% Also of interest may be the (relatively newer and more customizable)
% algorithmicx.sty package by Szasz Janos:
% http://www.ctan.org/tex-archive/macros/latex/contrib/algorithmicx/




% *** ALIGNMENT PACKAGES ***
%
%\usepackage{array}
% Frank Mittelbach's and David Carlisle's array.sty patches and improves
% the standard LaTeX2e array and tabular environments to provide better
% appearance and additional user controls. As the default LaTeX2e table
% generation code is lacking to the point of almost being broken with
% respect to the quality of the end results, all users are strongly
% advised to use an enhanced (at the very least that provided by array.sty)
% set of table tools. array.sty is already installed on most systems. The
% latest version and documentation can be obtained at:
% http://www.ctan.org/tex-archive/macros/latex/required/tools/


% IEEEtran contains the IEEEeqnarray family of commands that can be used to
% generate multiline equations as well as matrices, tables, etc., of high
% quality.




% *** SUBFIGURE PACKAGES ***
%\ifCLASSOPTIONcompsoc
%  \usepackage[caption=false,font=normalsize,labelfont=sf,textfont=sf]{subfig}
%\else
%  \usepackage[caption=false,font=footnotesize]{subfig}
%\fi
% subfig.sty, written by Steven Douglas Cochran, is the modern replacement
% for subfigure.sty, the latter of which is no longer maintained and is
% incompatible with some LaTeX packages including fixltx2e. However,
% subfig.sty requires and automatically loads Axel Sommerfeldt's caption.sty
% which will override IEEEtran.cls' handling of captions and this will result
% in non-IEEE style figure/table captions. To prevent this problem, be sure
% and invoke subfig.sty's "caption=false" package option (available since
% subfig.sty version 1.3, 2005/06/28) as this is will preserve IEEEtran.cls
% handling of captions.
% Note that the Computer Society format requires a larger sans serif font
% than the serif footnote size font used in traditional IEEE formatting
% and thus the need to invoke different subfig.sty package options depending
% on whether compsoc mode has been enabled.
%
% The latest version and documentation of subfig.sty can be obtained at:
% http://www.ctan.org/tex-archive/macros/latex/contrib/subfig/




% *** FLOAT PACKAGES ***
%
%\usepackage{fixltx2e}
% fixltx2e, the successor to the earlier fix2col.sty, was written by
% Frank Mittelbach and David Carlisle. This package corrects a few problems
% in the LaTeX2e kernel, the most notable of which is that in current
% LaTeX2e releases, the ordering of single and double column floats is not
% guaranteed to be preserved. Thus, an unpatched LaTeX2e can allow a
% single column figure to be placed prior to an earlier double column
% figure. The latest version and documentation can be found at:
% http://www.ctan.org/tex-archive/macros/latex/base/


%\usepackage{stfloats}
% stfloats.sty was written by Sigitas Tolusis. This package gives LaTeX2e
% the ability to do double column floats at the bottom of the page as well
% as the top. (e.g., "\begin{figure*}[!b]" is not normally possible in
% LaTeX2e). It also provides a command:
%\fnbelowfloat
% to enable the placement of footnotes below bottom floats (the standard
% LaTeX2e kernel puts them above bottom floats). This is an invasive package
% which rewrites many portions of the LaTeX2e float routines. It may not work
% with other packages that modify the LaTeX2e float routines. The latest
% version and documentation can be obtained at:
% http://www.ctan.org/tex-archive/macros/latex/contrib/sttools/
% Do not use the stfloats baselinefloat ability as IEEE does not allow
% \baselineskip to stretch. Authors submitting work to the IEEE should note
% that IEEE rarely uses double column equations and that authors should try
% to avoid such use. Do not be tempted to use the cuted.sty or midfloat.sty
% packages (also by Sigitas Tolusis) as IEEE does not format its papers in
% such ways.
% Do not attempt to use stfloats with fixltx2e as they are incompatible.
% Instead, use Morten Hogholm'a dblfloatfix which combines the features
% of both fixltx2e and stfloats:
%
% \usepackage{dblfloatfix}
% The latest version can be found at:
% http://www.ctan.org/tex-archive/macros/latex/contrib/dblfloatfix/




%\ifCLASSOPTIONcaptionsoff
%  \usepackage[nomarkers]{endfloat}
% \let\MYoriglatexcaption\caption
% \renewcommand{\caption}[2][\relax]{\MYoriglatexcaption[#2]{#2}}
%\fi
% endfloat.sty was written by James Darrell McCauley, Jeff Goldberg and 
% Axel Sommerfeldt. This package may be useful when used in conjunction with 
% IEEEtran.cls'  captionsoff option. Some IEEE journals/societies require that
% submissions have lists of figures/tables at the end of the paper and that
% figures/tables without any captions are placed on a page by themselves at
% the end of the document. If needed, the draftcls IEEEtran class option or
% \CLASSINPUTbaselinestretch interface can be used to increase the line
% spacing as well. Be sure and use the nomarkers option of endfloat to
% prevent endfloat from "marking" where the figures would have been placed
% in the text. The two hack lines of code above are a slight modification of
% that suggested by in the endfloat docs (section 8.4.1) to ensure that
% the full captions always appear in the list of figures/tables - even if
% the user used the short optional argument of \caption[]{}.
% IEEE papers do not typically make use of \caption[]'s optional argument,
% so this should not be an issue. A similar trick can be used to disable
% captions of packages such as subfig.sty that lack options to turn off
% the subcaptions:
% For subfig.sty:
% \let\MYorigsubfloat\subfloat
% \renewcommand{\subfloat}[2][\relax]{\MYorigsubfloat[]{#2}}
% However, the above trick will not work if both optional arguments of
% the \subfloat command are used. Furthermore, there needs to be a
% description of each subfigure *somewhere* and endfloat does not add
% subfigure captions to its list of figures. Thus, the best approach is to
% avoid the use of subfigure captions (many IEEE journals avoid them anyway)
% and instead reference/explain all the subfigures within the main caption.
% The latest version of endfloat.sty and its documentation can obtained at:
% http://www.ctan.org/tex-archive/macros/latex/contrib/endfloat/
%
% The IEEEtran \ifCLASSOPTIONcaptionsoff conditional can also be used
% later in the document, say, to conditionally put the References on a 
% page by themselves.




% *** PDF, URL AND HYPERLINK PACKAGES ***
%
%\usepackage{url}
% url.sty was written by Donald Arseneau. It provides better support for
% handling and breaking URLs. url.sty is already installed on most LaTeX
% systems. The latest version and documentation can be obtained at:
% http://www.ctan.org/tex-archive/macros/latex/contrib/url/
% Basically, \url{my_url_here}.





% *** Do not adjust lengths that control margins, column widths, etc. ***
% *** Do not use packages that alter fonts (such as pslatex).         ***
% There should be no need to do such things with IEEEtran.cls V1.6 and later.
% (Unless specifically asked to do so by the journal or conference you plan
% to submit to, of course. )


\usepackage{emp}
\ifx\pdftexversion\undefined
\usepackage[dvips]{graphicx}
\else
\usepackage[pdftex]{graphicx}
\DeclareGraphicsRule{*}{mps}{*}{}
\fi

\usepackage{listings}
\usepackage{caption}
\usepackage{color}
\usepackage{csquotes}
\MakeOuterQuote{"}
\definecolor{dkgreen}{rgb}{0,0.6,0}
\definecolor{gray}{rgb}{0.5,0.5,0.5}
\definecolor{mauve}{rgb}{0.58,0,0.82}
\usepackage{array}
\newcolumntype{L}[1]{>{\raggedright\let\newline\\\arraybackslash\hspace{0pt}}m{#1}}
\newcolumntype{C}[1]{>{\centering\let\newline\\\arraybackslash\hspace{0pt}}m{#1}}
\newcolumntype{R}[1]{>{\raggedleft\let\newline\\\arraybackslash\hspace{0pt}}m{#1}}
\lstset{frame=tb,
  language=C++,
  aboveskip=5mm,
  belowskip=5mm,
  columns=flexible,
  showstringspaces=false,
  basicstyle=\fontsize{8}{12}\ttfamily,
  numbers=none,
  numberstyle=\tiny\color{gray},
  keywordstyle=\color{blue},
  commentstyle=\color{dkgreen},
  stringstyle=\color{mauve},
  captionpos=b, 
  breaklines=true,
  breakatwhitespace=true
  tabsize=4
}

% correct bad hyphenation here
\hyphenation{op-tical net-works semi-conduc-tor}


\begin{document}
\begin{empfile}
\begin{empcmds}
input metauml;
\end{empcmds}
%
% paper title
% can use linebreaks \\ within to get better formatting as desired
% Do not put math or special symbols in the title.
\title{Statistical analysis of SCADA network communications}
%
%


\author{Francis~Deslauriers, Antoine~Lemay, Étienne~Ducharme, José~Fernandez% <-this % stops a space
\IEEEcompsocitemizethanks{\IEEEcompsocthanksitem F. Deslauriers is a undergraduate computer engineering student at Polytechnique Montréal, Montréal, Qc\protect\\
% note need leading \protect in front of \\ to get a newline within \thanks as
% \\ is fragile and will error, could use \hfil\break instead.
E-mail: francis.deslauriers@polymtl.ca
}% <-this % stops an unwanted space
%\thanks{Manuscript received April 19, 2005; revised December 27, 2012.}
}

%
\IEEEtitleabstractindextext{%
\begin{abstract}
 Applying notions from intrusion detection system, we built a tool named \texttt{ScadaAnalyzer} that learns from a network usual traffic and can determine if specific capture fits with the model distribution. Our SCADA packet analyzer takes advantage of the highly periodic behavior of the DNP3 protocol to detect unusual communication between hosts. This tool uses the Kolmogorov-Smirnov statistical test to compare a model distribution and a test sample. Gathering positive preliminary results during test runs suggest a safer future for SCADA computer security.
\end{abstract}

% Note that keywords are not normally used for peerreview papers.
\begin{IEEEkeywords}
Traffic Analysis, SCADA, Intrusion Detection System.
\end{IEEEkeywords}}


% make the title area
\maketitle


\IEEEdisplaynontitleabstractindextext

\IEEEpeerreviewmaketitle



\section{Introduction}
The electrical grid is a set of highly critical equipment spread across a large geographical area and therefore is complex to manage efficiently. With the increasing use of computer-controled devices on the grid, operators can monitor the state of the network from remote location. For example, operators can query for the state of a particular sensor on the network or send request to change its state without having a direct interaction with the device.

With the automation of the Grid comes the cyber-threats on the control network. The lack of suffisiant security on those networks would put a large number of people at risk. In cold countries, a failure of the electrical grid which powers most of the heating would have catastrophic consequences.

This article introduce \texttt{ScadaAnalyzer}, a tool developed to detect abnormal interactions between hosts on a control network. Having a small amount of noise on such network makes statistical analysis a posibility. This tool uses features from control network protocol to perform statistical analysis in order to detect intrusions and infected hosts.

\section{Related work}
This section includes a brief introduction to SCADA networks and explains the challenges that the industry is facing regarding computer security. It also dicuss how specific features of SCADA networks can be used to detect threats and anomalies.
\subsection{SCADA Networks}
SCADA networks are cyber physical networks designed for Supervisory Control And Data Acquisition. This type of network is used to remotely control industrial systems such as circuit breakers, valves or robots\cite{lemay}. Operators can have a real-time overview of the state of the network in order to adjust and react to measurements and alerts from all across the network. Critical control networks such as the electrical power grid are controlled and monitored by such system. An important advantage of using control networks consists of being able to affect the state of the network remotely. Also, the use of TCP/IP stack over the internet means that the power flow can be controlled and monitored without physical access to the components\cite{lemay}. Typically, a SCADA is formed by multiple devices spread all accross the network forming a logical tree. The root of this tree is the Master Terminal Unit (MTU) where all data and commands are sent to and from. MTUs are in charge of a large geographical section of the grid and use polling techniques to gather data from the leaves of that logical tree. The Remote Terminal Units (RTU) are the field components that retrieve the data from the sensors and forward it to their MTU.

A major drawback of using TCP/IP stack on this type of network is that the well-known internet vulnerabilities are easily transferable to SCADA networks context. An attacker controlling even a single host can have major impacts on populations. A good example of what is at stake when we talk about security on a SCADA network is the worm unveiled in 2010 by Symantec, Stuxnet \cite{symantec:stuxnet}. Stuxnet was specially designed to target Industrial Control Systems(ICS) in infrastructures such as power plants. Intrusion Detection Systems have an important role in these cases, but those tools may be hard to transfer to SCADA networks because of the highly specialized attackers targeting control networks. It is important to realize that traditionnal reaction to intrusion may not be applicable to highly critical networks. For exemple, an operator can not shutdown the electrical distribution network without having a major impact on the customers powered by this network.

There are multiple protocols used for communication between devices on a control network. An important protocol used in North American SCADA networks is Distributed Network Protocol version 3 (DNP3)\cite{lemay}. We focused our effort on this protocol for our analysis and experimentation. The use of this particular protocol has an important impact on the traffic. For example, the master-slave feature of this protocol has a noticeable effect on the topology of the communications \cite{clark}. Also, DNP3 use periodic pooling from the masters as well as exceptions reporting from the slaves\cite{lemay}. 

\subsection{Feature detection}
\label{subsectionFeature}
As mentionned earlier, we focused our analysis on the DNP3 network protocol but most of the ideas used here also apply to other control protocols such as Modbus. The use of the DNP3 protocol affects the way hosts on a network interact with each other. As Lemay \cite{lemay} has shown in his thesis, those interactions occur in a very regular fashion. Also the fact that the DNP3 protocol makes use of a slave-master hierarchy is really specific to control networks. These two important characteristics help us with the fact that unusual behavior is easily detectable on this type of network. Here we will explain how we used three features of SCADA networks to detect possible cyber-attackers or infected hosts. In this section, we will discuss the features that Lemay studied.

\subsubsection{Topology}
On most SCADA networks, the server-client model does not stand. As mentioned before, DNP3 protocol works on a Master-Slave hierarchy. This feature of SCADA networks makes it really easy to detect abnormal behaviour. In really few occasions will an RTU initiate a conversation with a MTU without being previously polled. It would be really suspicious behavior for a RTU to send a large number of packets to the MTU, other RTUs and even hosts outside the network. This makes the network topology a good feature to test in order to detect infected units in a control network.

\subsubsection{Packet Length}
Another effect of the polling characteristic of most SCADA networks is that there is a really low variation of the packet size. When polling for data from the sensors, the packet containing the request is likely to be the same each time and the same applies to the response from the RTU. Recording an important change in the size of the packets travelling on the network would be a good indicator that an host is compromised.


\subsubsection{Interdeparture Time}
The last feature of SCADA control networks that was considered during this experimentation is the interdeparture time of packets from a particular host. We are still basing our thinking on the regularity mentioned earlier. Since control networks are most of the time used for data gathering and command sending to the RTUs the departure pattern is quite regular. For example, when polling a particular sensor a MTU will first send a polling request then receive the response and finally send an acknowledgement. The inter-departure time between the request and the acknowledgement will be stable over time since network has a really low traffic. Also, the fact that polling is done at a particular frequency makes this feature easily detectable when comparing two network packet capture.
\begin{figure}[ht!]
\centering
\includegraphics[width=3.2in]{interdep_dist.png}
\caption{Average interdepature time from RTU \cite{lemay}}
\label{fig:interdep}
\end{figure}
Figure ~\ref{fig:interdep} from Lemay's Ph.D. thesis is showing the average interdepature time distribution from the RTUs. We can clearly see that 50\% of the interdepature time is distributed between about three timings.

\section{Model}
As we explained in section ~\ref{subsectionFeature}, some patterns of communication between hosts on a DNP3.0 network are happening in a really tight distribution. It is this distribution shape that was used in our tool to detect threat on the network. In this section, we will explained how we understood the problem and how we solved it. 

\subsection{Challenge}
With that tool we are trying to automate the detection of the distribution of specific features that are outside the regular state of the network. As we saw from the work of Lemay, the distributions are far from the normal distribution which means that we had to find a statistical test that does not make any assumption on the shape of the distribution. This tool must first learn how the hosts on the network normally behave and it is not possible to use a model from another network because the technical configuration might be different from a network to another. 
We also want to create a tool that can easily be upgraded and adapted to the situation. According to the network, features must be easily added, modified or removed.
\subsection{Solving the problem}
While analyzing the problem, it was decided that our program should have two operation modes. First of all, there is the Learning mode. The learning mode is when the program is recording every interaction between the hosts of the network. This model is then stored as a Json file and will be considered as usual pattern for the analysis stage. Next, there is the Analyzing mode. Running in this mode compare the packet capture against the distributions recorded during the learning stage. It is possible to learn and analyze with a live capture directly from the network interface card. It is also possible to use those two modes with a packet capture file that has been previously recorded on the network. Finally, a key element of our tool was that it must be possible for the user to specify a critical value for the statistical test. If not specified, a default value is used.

In this tool, we used the Kolmogorov Smirnov test (KS test) that is test for goodness of fit. As Masset \cite{kstest} discussed a test for goodness of fit ''is based on the maximum difference between an empirical and a hypothetical cumulative distribution''. In our case, we will use the term model distribution to talk about the hypothetical distribution. 
On the other side, the topology feature test confirms that specific topology has been seen in the model distribution without any statistical evaluation. We will run this test on each feature independantly. We want our tool to offer a good modularity and to be ready for the addition of needed feature handlers.

\label{subsectionStat}
\section{Implementation}
\subsection{Software structure}
\label{impl}
The idea behind our design was modularity. We wished to be able to add new test handlers as we want. During our design phase we decided to use the \verb!C++! programming language because of its powerful object oriented features and also because of our experience. In figure~\ref{fig:classdiagramm} you can see our three test handler classes that inherit from the FeatureTestHandler class. 
\begin{figure}[!ht]
\centering
\begin{emp}[classdiag](100,100)
Class.A("FeatureTstHdlr")()();
Class.B("TopologyTestHandlr")()();
Class.C("PacketLengthTestHdlr")()();
Class.D("InterDepartureTestHdlr")()();
Class.E("TstHdlrContainer")()();
B.e = A.e + (0, -120);
C.e = B.n + (-3, 20);
D.e = C.n + (-3, 20);


rightToLeft(85)(A,E);
drawObjects(A,B,C,D, E);

link(aggregation)(E.e -- A.w);
clink(inheritance)(B, A);
clink(inheritance)(C,A);
clink(inheritance)(D, A);

item(iAssoc)("Container")(obj.sw = E.e);
item(iAssoc)("1..1")(obj.nw = E.e);

item(iAssoc)("TestHandler")(obj.se = A.w+(0,4));
item(iAssoc)("0..*")(obj.ne = A.w +(0, -4));


\end{emp}

\caption{Class diagram of the test Handlers and the container}
\label{fig:classdiagramm}
\end{figure}
The next important part is the TestHandlerContainer class that is keeping a vector FeatureTestHandler objects to use. Every test handler classes implements the FeatureTestHandler virtual interface that is shown by listing ~\ref{lst:ftrTestHdlr}.

\begin{lstlisting}[caption=FeatureTestHandler class virtual interface, label=lst:ftrTestHdlr]
virtual void JsonToData(Json::Value *json) = 0;
virtual void computePacket(
                  const struct pcap_pkthdr* pkthdr,
                  const unsigned char * packet ) = 0;
virtual void printDistribution() const = 0;
virtual void runTest( double cAlpha) = 0;
virtual int getTestResult() = 0;
virtual Json::Value *DataToJson() const = 0;
\end{lstlisting}

We will walkthrough this interface  and explain the use of these six methods. Some of these methods may be necessary during learning and/or analysis mode. Also, it is important to note that the TestHandlerContainer class calls these methods for each handler. 

First of all, the implementation of the JsonToData method is looking for model data for a particular test handler in the model file. Each implementation of the FeatureTestHandler interface knows how to extract this data for its own feature.

Next, the computePacket method is called for each packet in the capture for each handler in the test. Each handler is aware of how to consider each packet. This method takes in argument pointers to packet header and packet structures.

Moreover, the printDistribution method is primarily for testing purposes. It is not used in production code but is really helpful during development phase.

Next, the runTest method is running the actual statistical test that will be discussed in the subsection ~\ref{subsectionStat}. It takes the critical value that is to be used in that test as argument. Again, each handler is aware of how to run the test for its particular feature. 

The getTestResult method then returns the results of the statistical test. In its current state, the tool is simply printing which tests have failed or passed. It would be really easy to add some other action to be taken of failed test runs. For example, we could easily send an email stating which test has failed.

Finally, the DataToJson method is to save the model to the model file after a learning session. Each implementation returns a Json document to the TestHandlerContainer which is saved in the file.

\subsection{Tools \& Libraries}
This tool has been implemented in C++11. It is taking advantage of the many object-oriented features of this language. This software was designed with polymorphism in mind which positively impacted its modularity. In fact, it is really easy to add new features to detect and test. The user only has to implement the virtual interface defined by the FeatureTestHandler and add a newly allocated object in the TestHandlerContainer object.

Traffic analysis for security purposes has been used for several years and many tools are free available out there. When building a tool like the ScadaAnalyzer, it is important to take advantage of the work others have open sourced. This program was built using the free and open source libpcap library to help with the online and offline packet parsing\cite{libpcap}. 

A key aspect of this tool is that it records a model of what is normal on the network. In order to save this model, we use a Json file called the \texttt{model.json}. The JSON parsing and writing of this file is done with the help of JSONcpp \verb!C++! library\cite{jsoncpp}. It is important to note that this library is licensed under a Public domain license.

Finally, we use the C++ Boost library for file system operations\cite{boost}.
\section{Performance analysis}
In this section, we explain our experiental methodology for testing the efficiency of our tool. We are also exposing the results of those tests.
\subsection{Methodology}
In order to test the suitability of our tool, we used a packet capture recorded from a simulated network. This traffic was generated in an ICS sandbox design by LEMAY \cite{lemay}. The testbench was made of seven virtual machines running in VMware Workstation software. As shown in figure ~\ref{fig:network}, six of these virtual machines were representating RTUs and one was the MTU. The use of a simulated network instead of a real world dataset was motivated by the limited availability of production traffic capture and the fact that we wanted to simulate real world malware. 
For our learning phase, we used packet capture from  normal interactions on this simulated network. In order to simulated an infected host on the network, we later installed the Waledac malware on host named RLS103. 
This is the setup that we used in our the analysis phase. Two scenarios were executed in order to test \texttt{ScadaAnalyzer}. Firstly, the Waledac scenario where a RTU is infected with the Waledac worm. Secondly, the Advance Persistent Threat is mimicking an intrusion in a RTU by an attacker.
It is important to note that we performed our tests with a confidence interval of 0.05.

\begin{figure}[ht!]
\centering
\includegraphics[width=3in]{network1.png}
\caption{Testbench network map \cite{lemay}}
\label{fig:network}
\end{figure}


\subsection{Results}
\subsubsection{Topology}
\label{sec:topology}
In table ~\ref{tab:topo} shows the result of the topology test. Those are the interaction that our tool has not encountered in the model distribution. The source and destination columns are representing the source and destination hosts respectively.
\begin{table}[ht!]
\centering
\captionof{table}{Unknown interaction detected} \label{tab:topo} 

    \begin{tabular}{ | C{2cm}| C{2.5cm} | C{2.5cm} |}
    \hline
      \multicolumn{2}{|c|}{\bf Source} & \bf Destination \\ \hline 
    RSL103 &172.31.255.103 &    89.18.58.10 \\[0.2cm] \hline
    RLS103 & 172.31.255.103 &  117.102.35.90 \\[0.2cm] \hline
    RLS103 & 172.31.255.103 & 69.203.207.115 \\[0.2cm] \hline
    RLS103 & 172.31.255.103 &  83.87.159.131 \\[0.2cm] \hline
    RLS103 & 172.31.255.103 & 119.192.145.145 \\ [0.2cm]\hline

    \end{tabular}
\end{table}

\subsubsection{Packet size}
Table~\ref{tab:size} exposes the results of the packet size tests with two different samples. The dataset column represents the name of the scenario simulated in our custom sandbox. The Status column shows if the sample data matches with the model data. The status is compute from the Dstat and Critical Value columns. The column titled DStat represents the largest difference between the cumulative distributions value of the test and the model samples. The critical value is compute from the confidence interval, the size of the test sample and the size of the model sample. 
\begin{table}[ht!]
\centering
\captionof{table}{Packet size} \label{tab:size} 

    \begin{tabular}{ | C{2.4cm} | C{1.1cm} |C{1cm} | C{2cm}  |}
    \hline
        \bf Dataset & \bf Status & \bf Dstat & \bf Critical value ($\alpha$ = 0.05)\\ \hline 
          Waledac & Failed & 0.032 & 0.026\\ [0.2cm]\hline
          Advance Persistent Threat (APT)& Failed & 0.411 & 0.013 \\ \hline
    \end{tabular}
\end{table}
First, the Waledac dataset is the scenario shown in section~\ref{sec:topology} where an host has been infected by the Waledac worm. Secondly, the Advance Persistent Threat is a scenario where a cyber-attacker has been able to take control of a RTU in the network. From this infected node, he is now trying to expand is control to other RTUs in the network using the tool meterpreter. 


\subsubsection{Interdeparture time}
Table~\ref{tab:interdep} shows the results of the interdeparture time test run. The column names have identical meaning that in table~\ref{tab:size}. This data was gathered testing the Waledac scenario against the model distribution.

\begin{table}[ht!]
\centering
\captionof{table}{Interdeparture Time} \label{tab:interdep} 

    \begin{tabular}{|C{1.1cm} | C{2.1cm} | C{1cm} |C{0.8cm} | C{1.1cm} |}
    \hline
        \multicolumn{2}{|c|}{\bf Host} & \bf Status & \bf Dstat & \bf Critical value ($\alpha$=0.05) \\ \hline 
        MTU & 172.31.255.100 & Failed & 0.052 & 0.035 \\[0.2cm] \hline
        RLS103 & 172.31.255.103 & Failed & 0.124 & 0.091 \\[0.2cm]  \hline
        RLS104 & 172.31.255.104 & Passed & 0.053 & 0.095 \\[0.2cm]  \hline
        RLS105 & 172.31.255.105 & Passed & 0.081 & 0.094 \\[0.2cm]  \hline
        RLS106 &  172.31.255.106 & Passed & 0.062 & 0.099 \\[0.2cm]  \hline
        RLS107 &  172.31.255.107 & Passed & 0.055 & 0.099 \\[0.2cm]  \hline

    \end{tabular}
\end{table}

\section{Discussion}
%topology
First, we can see from the topology test in table~\ref{tab:topo} that we detected an host at IP address 172.31.255.103 that is trying to send packets to other hosts that are not in the subnet. It is most likely the Waledac worm trying to reach its Command and Control (C\&C) server. As Lemay explained, this worm has a list of peer that it needs to query in order to notify the C\&C of its presence \cite{lemay}. This type of behavior is easy to detect, specially when a RTU is trying to reach addresses outside of the network. Moreover, the IP addresses that RLS103 is trying to reach are spread across three continents: North America, Europe and Asia.
%packet size

Secondly, we can see in table~\ref{tab:size} that in our test with the Waledac scenario there is a 0.032 gap and that 0.026 was the critical value for that sample. Also, we can see that the Advance Persistent Threat (APT) scenario is 0.399 points out of the critical value. In both of those tests we have effectively detected the threat. From this data, it is possible see that these type of abnormal behavior can easily be detected. Those results match with those that Lemay has gathered. This easily detectable traffic is most likely due to the fact that the worm Waledac is trying to reach its C\&C server with no consideration of the type of network it is in. In the same line, meterpreter software that is used in the APT scenario is trying, by default to send exploits and payload as fast as possible which would unsuual packet size.

%interdeparture
Finally, we can see from table~\ref{tab:interdep} that the distribution of the interdeparture time for the RLS103 RTU is statistically different from the model distribution. The difference is high enough to reject the null hypothesis which indicates a potential threat to the network. Also, the MTU has also failed the test with a 0.052 gap with the model which is an unexpected result. This can be explained by the incomplete model dataset. A packet capture on a longer period would help having a better idea of the of the normal behavior.
% An example of a floating figure using the graphicx package.
% Note that \label must occur AFTER (or within) \caption.
% For figures, \caption should occur after the \includegraphics.
% Note that IEEEtran v1.7 and later has special internal code that
% is designed to preserve the operation of \label within \caption
% even when the captionsoff option is in effect. However, because
% of issues like this, it may be the safest practice to put all your
% \label just after \caption rather than within \caption{}.
%
% Reminder: the "draftcls" or "draftclsnofoot", not "draft", class
% option should be used if it is desired that the figures are to be
% displayed while in draft mode.
%
%\begin{figure}[!t]
%\centering
%\includegraphics[width=2.5in]{myfigure}
% where an .eps filename suffix will be assumed under latex, 
% and a .pdf suffix will be assumed for pdflatex; or what has been declared
% via \DeclareGraphicsExtensions.
%\caption{Simulation Results.}
%\label{fig_sim}
%\end{figure}

% Note that IEEE typically puts floats only at the top, even when this
% results in a large percentage of a column being occupied by floats.
% However, the Computer Society has been known to put floats at the bottom.


% An example of a double column floating figure using two subfigures.
% (The subfig.sty package must be loaded for this to work.)
% The subfigure \label commands are set within each subfloat command,
% and the \label for the overall figure must come after \caption.
% \hfil is used as a separator to get equal spacing.
% Watch out that the combined width of all the subfigures on a 
% line do not exceed the text width or a line break will occur.
%
%\begin{figure*}[!t]
%\centering
%\subfloat[Case I]{\includegraphics[width=2.5in]{box}%
%\label{fig_first_case}}
%\hfil
%\subfloat[Case II]{\includegraphics[width=2.5in]{box}%
%\label{fig_second_case}}
%\caption{Simulation results.}
%\label{fig_sim}
%\end{figure*}
%
% Note that often IEEE papers with subfigures do not employ subfigure
% captions (using the optional argument to \subfloat[]), but instead will
% reference/describe all of them (a), (b), etc., within the main caption.


% An example of a floating table. Note that, for IEEE style tables, the 
% \caption command should come BEFORE the table. Table text will default to
% \footnotesize as IEEE normally uses this smaller font for tables.
% The \label must come after \caption as always.
%
%\begin{table}[!t]
%% increase table row spacing, adjust to taste
%\renewcommand{\arraystretch}{1.3}
% if using array.sty, it might be a good idea to tweak the value of
% \extrarowheight as needed to properly center the text within the cells
%\caption{An Example of a Table}
%\label{table_example}
%\centering
%% Some packages, such as MDW tools, offer better commands for making tables
%% than the plain LaTeX2e tabular which is used here.
%\begin{tabular}{|c||c|}
%\hline
%One & Two\\
%\hline
%Three & Four\\
%\hline
%\end{tabular}
%\end{table}


% Note that IEEE does not put floats in the very first column - or typically
% anywhere on the first page for that matter. Also, in-text middle ("here")
% positioning is not used. Most IEEE journals use top floats exclusively.
% However, Computer Society journals sometimes do use bottom floats - bear
% this in mind when choosing appropriate optional arguments for the
% figure/table environments.
% Note that, LaTeX2e, unlike IEEE journals, places footnotes above bottom
% floats. This can be corrected via the \fnbelowfloat command of the
% stfloats package.



\section{Conclusion}
This article presented a SCADA traffic statistical analysis tool that can be used onsite or offsite to detect cyber-threats. We have successfully detected threats using features of SCADA control networks. We also had a false positive during our test runs. This false positive is probably due the limited size of the packet capture during the learning phase. Moreover, usual tools and malware were shown to be easily detectable because of the default behavior that is likely to be motivated by ''a faster infection is a better infection''. Which means that a cyber-attacker would need to put a lot of effort to modify existing tools to minic the way hosts are interacting on this specific network.

Further work could include expanding the tool with more features. For example, it would be interesting to develop a test handler that would specially detect if an host is trying to reach and address outside the control network subnet. We are already detecting this type of behavior with the topology handler but it would be useful to set a specific way to react to that really alarming event. Also, it would be interesting to specify different critical value for each test. It is easy to picture an use-case where the user would like a specific feature to be more sensible to a variation of distribution.


% if have a single appendix:
%\appendix[Proof of the Zonklar Equations]
% or
%\appendix  % for no appendix heading
% do not use \section anymore after \appendix, only \section*
% is possibly needed

% use appendices with more than one appendix
% then use \section to start each appendix
% you must declare a \section before using any
% \subsection or using \label (\appendices by itself
% starts a section numbered zero.)
%


\appendices

% you can choose not to have a title for an appendix
% if you want by leaving the argument blank


% Can use something like this to put references on a page
% by themselves when using endfloat and the captionsoff option.
\ifCLASSOPTIONcaptionsoff
  \newpage
\fi



% trigger a \newpage just before the given reference
% number - used to balance the columns on the last page
% adjust value as needed - may need to be readjusted if
% the document is modified later
%\IEEEtriggeratref{8}
% The "triggered" command can be changed if desired:
%\IEEEtriggercmd{\enlargethispage{-5in}}

% references section

% can use a bibliography generated by BibTeX as a .bbl file
% BibTeX documentation can be easily obtained at:
% http://www.ctan.org/tex-archive/biblio/bibtex/contrib/doc/
% The IEEEtran BibTeX style support page is at:
% http://www.michaelshell.org/tex/ieeetran/bibtex/
%\bibliographystyle{IEEEtran}
% argument is your BibTeX string definitions and bibliography database(s)
%\bibliography{IEEEabrv,../bib/paper}
%
% <OR> manually copy in the resultant .bbl file
% set second argument of \begin to the number of references
% (used to reserve space for the reference number labels box)
\begin{thebibliography}{1}

\bibitem{lemay}
A.~Lemay, \emph{Defending the SCADA Network Controlling the Electrical Grid from Advanced Persistent Threats (Ph.D. Thesis)},\hskip 1em plus
  0.5em minus 0.4em\relax Polytechnique Montréal, Québec, Canada, 2013.
\bibitem{symantec:stuxnet}
N. Falliere, L. O. Murchu and E. Chien, \emph{W32.Stuxnet Dossier Version 1.4},\hskip 1em plus 0.5em minus 0.4em\relax Symantec
Security Response, 2011.  
\bibitem{clark}
G. Clark and D. Reynders, \emph{ Pratcical Modern SCADA Protocols: DNP3, IEC 60870.5 and Related Systems},\hskip 1em plus 0.5em minus 0.4em\relax China: Newnes (Elsevier), 2008.

\bibitem{jsoncpp}
Baptiste Lepilleur, \emph{ JSONcpp},\hskip 1em plus 0.5em minus 0.4em\relax \url{http://jsoncpp.sourceforge.net/}

\bibitem{kstest}
Frank J. Massey, Jr.\emph{The Kolmogorov-Smirnov Test for Goodness of Fit}\hskip 1em plus 0.5em minus 0.4em\relax Journal of the American Statistical Association
Vol. 46, No. 253 (Mar., 1951), pp. 68-78

\bibitem{boost}
Boost project, \emph{Boost C++ Library},\hskip 1em plus 0.5em minus 0.4em\relax \url{http://www.boost.org/doc/libs/1_55_0/libs/filesystem/doc/index.htm}

\bibitem{libpcap}
LibPcap, \emph{libpcap, a portable C/C++ library for network traffic capture.},\hskip 1em plus 0.5em minus 0.4em\relax \url{http://www.tcpdump.org/}

\end{thebibliography}

% You can push biographies down or up by placing
% a \vfill before or after them. The appropriate
% use of \vfill depends on what kind of text is
% on the last page and whether or not the columns
% are being equalized.

%\vfill

% Can be used to pull up biographies so that the bottom of the last one
% is flush with the other column.
%\enlargethispage{-5in}



% that's all folks
\end{empfile}
\end{document}

